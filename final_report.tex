% Options for packages loaded elsewhere
\PassOptionsToPackage{unicode}{hyperref}
\PassOptionsToPackage{hyphens}{url}
%
\documentclass[
]{article}
\title{Stats 140XP: Final Project}
\author{Ethan Allavarpu \(\cdot\) Raymond Bai \(\cdot\) Jaclyn Chiu
\(\cdot\) Ariel Chow \(\cdot\) Carlie Lin \(\cdot\) Dara
Tan \and \textbf{Explore the GLOBE}}
\date{1 December 2021}

\usepackage{amsmath,amssymb}
\usepackage{lmodern}
\usepackage{iftex}
\ifPDFTeX
  \usepackage[T1]{fontenc}
  \usepackage[utf8]{inputenc}
  \usepackage{textcomp} % provide euro and other symbols
\else % if luatex or xetex
  \usepackage{unicode-math}
  \defaultfontfeatures{Scale=MatchLowercase}
  \defaultfontfeatures[\rmfamily]{Ligatures=TeX,Scale=1}
\fi
% Use upquote if available, for straight quotes in verbatim environments
\IfFileExists{upquote.sty}{\usepackage{upquote}}{}
\IfFileExists{microtype.sty}{% use microtype if available
  \usepackage[]{microtype}
  \UseMicrotypeSet[protrusion]{basicmath} % disable protrusion for tt fonts
}{}
\makeatletter
\@ifundefined{KOMAClassName}{% if non-KOMA class
  \IfFileExists{parskip.sty}{%
    \usepackage{parskip}
  }{% else
    \setlength{\parindent}{0pt}
    \setlength{\parskip}{6pt plus 2pt minus 1pt}}
}{% if KOMA class
  \KOMAoptions{parskip=half}}
\makeatother
\usepackage{xcolor}
\IfFileExists{xurl.sty}{\usepackage{xurl}}{} % add URL line breaks if available
\IfFileExists{bookmark.sty}{\usepackage{bookmark}}{\usepackage{hyperref}}
\hypersetup{
  pdftitle={Stats 140XP: Final Project},
  pdfauthor={Ethan Allavarpu \textbackslash cdot Raymond Bai \textbackslash cdot Jaclyn Chiu \textbackslash cdot Ariel Chow \textbackslash cdot Carlie Lin \textbackslash cdot Dara Tan; },
  hidelinks,
  pdfcreator={LaTeX via pandoc}}
\urlstyle{same} % disable monospaced font for URLs
\usepackage[margin=1in]{geometry}
\usepackage{graphicx}
\makeatletter
\def\maxwidth{\ifdim\Gin@nat@width>\linewidth\linewidth\else\Gin@nat@width\fi}
\def\maxheight{\ifdim\Gin@nat@height>\textheight\textheight\else\Gin@nat@height\fi}
\makeatother
% Scale images if necessary, so that they will not overflow the page
% margins by default, and it is still possible to overwrite the defaults
% using explicit options in \includegraphics[width, height, ...]{}
\setkeys{Gin}{width=\maxwidth,height=\maxheight,keepaspectratio}
% Set default figure placement to htbp
\makeatletter
\def\fps@figure{htbp}
\makeatother
\setlength{\emergencystretch}{3em} % prevent overfull lines
\providecommand{\tightlist}{%
  \setlength{\itemsep}{0pt}\setlength{\parskip}{0pt}}
\setcounter{secnumdepth}{-\maxdimen} % remove section numbering
\usepackage{booktabs}
\usepackage{longtable}
\usepackage{array}
\usepackage{multirow}
\usepackage{wrapfig}
\usepackage{float}
\usepackage{colortbl}
\usepackage{pdflscape}
\usepackage{tabu}
\usepackage{threeparttable}
\usepackage{threeparttablex}
\usepackage[normalem]{ulem}
\usepackage{makecell}
\usepackage{xcolor}
\usepackage{booktabs}
\usepackage{longtable}
\usepackage{array}
\usepackage{multirow}
\usepackage{wrapfig}
\usepackage{float}
\usepackage{colortbl}
\usepackage{pdflscape}
\usepackage{tabu}
\usepackage{threeparttable}
\usepackage{threeparttablex}
\usepackage[normalem]{ulem}
\usepackage{makecell}
\usepackage{xcolor}
\ifLuaTeX
  \usepackage{selnolig}  % disable illegal ligatures
\fi

\begin{document}
\maketitle

{
\setcounter{tocdepth}{1}
\tableofcontents
}
\vfill

\newpage

\hypertarget{abstract}{%
\section{Abstract}\label{abstract}}

We wanted to answer two main questions: which leadership qualities do
countries tend to view similarly and if countries align their
perceptions societal practices and values. For determining leadership
qualities and similar countries, we used principal component analysis
(PCA) then \(k\)-means clustering to create four ``clusters'' of
countries with similar leadership beliefs. We used the \_\_\_\_ method
to \_\_\_\_. We found that \_\_\_\_. We also looked at \_\_\_\_\_. In
the future, we recommend looking into \_\_\_\_\_. Some limitations to
our project are \_\_\_\_\_\_.

\hypertarget{problem-statements}{%
\section{Problem Statements}\label{problem-statements}}

\begin{enumerate}
\def\labelenumi{\arabic{enumi}.}
\tightlist
\item
  Which characteristics or traits do countries tend to group together
  when determining ``good'' leadership values?
\end{enumerate}

\begin{itemize}
\tightlist
\item
  Which countries have similar perceptions of these leadership values?
\end{itemize}

\begin{enumerate}
\def\labelenumi{\arabic{enumi}.}
\setcounter{enumi}{1}
\tightlist
\item
  Do societal practices and societal values align?

  \begin{itemize}
  \tightlist
  \item
    If they do not, which practices and values deviate most
    significantly?
  \end{itemize}
\end{enumerate}

\hypertarget{description-of-dataset}{%
\section{Description of Dataset}\label{description-of-dataset}}

The data set we have chosen to analyze is the \textbf{Dana Landis
Leadership} dataset, which comes from the GLOBE Research Survey. The
data provided in the folder had survey results for (1) leadership and
(2) societal and culture data and a PDF describing the nature of the
survey, but nothing more. To glean more information, we found the two
questionnaires (alpha and beta) described in the informational PDF to
get the original questions asked in the surveys. While we do not have a
``codebook'' in a traditional sense, the original questions asked may
help guide us in understanding what each variable means and how the
survey represents respondents answers numerically. The survey is on a 1
to 7 scale, with 1 being a negative response, 4 a ``neutral'' score, and
7 positive.

Here is a look at 6 full and complete observations from the leadership
survey:

Here is a look at 6 full and complete observations from the social and
cultural survey:

\hypertarget{description-of-variables}{%
\section{Description of Variables}\label{description-of-variables}}

\hypertarget{visualization-and-exploratory-data-analysis}{%
\section{Visualization and Exploratory Data
Analysis}\label{visualization-and-exploratory-data-analysis}}

\hypertarget{analysis}{%
\section{Analysis}\label{analysis}}

\hypertarget{leadership-values}{%
\subsection{Leadership Values}\label{leadership-values}}

For the leadership values problem statement, the first objective is to
collapse the data into the first four principal components through
principal component analysis (PCA). PCA finds the directions which
capture the most variability in the data, so the first four account for
the maximum variation. PCA allows us to visualize trends in the
leadership values: countries tend to have similar sentiments about
variables that ``point'' in the same direction (had principal component
values that aligned).

Before performing PCA, though, we remove the second-order factor
analysis variables due to the heavy correlation with the original
predictor variables and a more difficult interpretation of these
variables. Since our goal is to understand the relationship between
certain leadership characteristics, keeping these complicated variables
might reduce our understanding of some characteristics.

After performing PCA, we visualize the directions of the first two
principal components. We also perform \(k\)-means clustering on the
first four principal components to determine the ``groups'' of
leadership characteristics that have similar perceptions:

\begin{center}\includegraphics[width=0.7\linewidth]{final_report_files/figure-latex/unnamed-chunk-3-1} \end{center}

\begin{center}\includegraphics[width=0.85\linewidth]{final_report_files/figure-latex/unnamed-chunk-4-1} \end{center}

After considering the leadership characteristics, we cluster countries
based on similar perceptions. We do this by using the variables of the
countries transformed into the first four principal components, then
running PCA on those components. The clustering results segregate the
countries into the following segments:

\begin{center}\includegraphics[width=0.85\linewidth]{final_report_files/figure-latex/unnamed-chunk-5-1} \end{center}
\begin{table}[h]

\caption{\label{tab:unnamed-chunk-6}Regions with Similar Leadership Perceptions}
\centering
\begin{tabular}[t]{>{}l>{}l>{}l>{}l}
\toprule
Cluster 1 & Cluster 2 & Cluster 3 & Cluster 4\\
\midrule
\textcolor[HTML]{739999}{\cellcolor{gray!6}{Costa Rica}} & \textcolor[HTML]{000026}{\cellcolor{gray!6}{India}} & \textcolor[HTML]{407326}{\cellcolor{gray!6}{Qatar}} & \textcolor[HTML]{BF7300}{\cellcolor{gray!6}{England}}\\
\textcolor[HTML]{739999}{Italy} & \textcolor[HTML]{000026}{Venezuela} & \textcolor[HTML]{407326}{Morocco} & \textcolor[HTML]{BF7300}{Namibia}\\
\textcolor[HTML]{739999}{\cellcolor{gray!6}{Ecuador}} & \textcolor[HTML]{000026}{\cellcolor{gray!6}{Taiwan}} & \textcolor[HTML]{407326}{\cellcolor{gray!6}{South Africa (Black Sample)}} & \textcolor[HTML]{BF7300}{\cellcolor{gray!6}{Czech Republic}}\\
\textcolor[HTML]{739999}{El Salvador} & \textcolor[HTML]{000026}{Hong Kong} & \textcolor[HTML]{407326}{France} & \textcolor[HTML]{BF7300}{Singapore}\\
\textcolor[HTML]{739999}{\cellcolor{gray!6}{Israel}} & \textcolor[HTML]{000026}{\cellcolor{gray!6}{Iran}} & \textcolor[HTML]{407326}{\cellcolor{gray!6}{}} & \textcolor[HTML]{BF7300}{\cellcolor{gray!6}{Kazakhstan}}\\
\addlinespace
\textcolor[HTML]{739999}{Hungary} & \textcolor[HTML]{000026}{Mexico} & \textcolor[HTML]{407326}{} & \textcolor[HTML]{BF7300}{Portugal}\\
\textcolor[HTML]{739999}{\cellcolor{gray!6}{Zambia}} & \textcolor[HTML]{000026}{\cellcolor{gray!6}{Russia}} & \textcolor[HTML]{407326}{\cellcolor{gray!6}{}} & \textcolor[HTML]{BF7300}{\cellcolor{gray!6}{Finland}}\\
\textcolor[HTML]{739999}{Zimbabwe} & \textcolor[HTML]{000026}{Indonesia} & \textcolor[HTML]{407326}{} & \textcolor[HTML]{BF7300}{Ireland}\\
\textcolor[HTML]{739999}{\cellcolor{gray!6}{Colombia}} & \textcolor[HTML]{000026}{\cellcolor{gray!6}{South Korea}} & \textcolor[HTML]{407326}{\cellcolor{gray!6}{}} & \textcolor[HTML]{BF7300}{\cellcolor{gray!6}{Austria}}\\
\textcolor[HTML]{739999}{Turkey} & \textcolor[HTML]{000026}{China} & \textcolor[HTML]{407326}{} & \textcolor[HTML]{BF7300}{Switzerland}\\
\addlinespace
\textcolor[HTML]{739999}{\cellcolor{gray!6}{Spain}} & \textcolor[HTML]{000026}{\cellcolor{gray!6}{Japan}} & \textcolor[HTML]{407326}{\cellcolor{gray!6}{}} & \textcolor[HTML]{BF7300}{\cellcolor{gray!6}{Netherlands}}\\
\textcolor[HTML]{739999}{Guatemala} & \textcolor[HTML]{000026}{Albania} & \textcolor[HTML]{407326}{} & \textcolor[HTML]{BF7300}{French Switzerland}\\
\textcolor[HTML]{739999}{\cellcolor{gray!6}{Bolivia}} & \textcolor[HTML]{000026}{\cellcolor{gray!6}{Poland}} & \textcolor[HTML]{407326}{\cellcolor{gray!6}{}} & \textcolor[HTML]{BF7300}{\cellcolor{gray!6}{Australia}}\\
\textcolor[HTML]{739999}{Greece} & \textcolor[HTML]{000026}{Egypt} & \textcolor[HTML]{407326}{} & \textcolor[HTML]{BF7300}{Sweden}\\
\textcolor[HTML]{739999}{\cellcolor{gray!6}{Brazil}} & \textcolor[HTML]{000026}{\cellcolor{gray!6}{Kuwait}} & \textcolor[HTML]{407326}{\cellcolor{gray!6}{}} & \textcolor[HTML]{BF7300}{\cellcolor{gray!6}{South Africa (White Sample)}}\\
\addlinespace
\textcolor[HTML]{739999}{Philippines} & \textcolor[HTML]{000026}{Nigeria} & \textcolor[HTML]{407326}{} & \textcolor[HTML]{BF7300}{Canada (English-speaking)}\\
\textcolor[HTML]{739999}{\cellcolor{gray!6}{Argentina}} & \textcolor[HTML]{000026}{\cellcolor{gray!6}{Malaysia}} & \textcolor[HTML]{407326}{\cellcolor{gray!6}{}} & \textcolor[HTML]{BF7300}{\cellcolor{gray!6}{New Zealand}}\\
\textcolor[HTML]{739999}{Slovenia} & \textcolor[HTML]{000026}{Georgia} & \textcolor[HTML]{407326}{} & \textcolor[HTML]{BF7300}{Germany (EAST)}\\
\textcolor[HTML]{739999}{\cellcolor{gray!6}{}} & \textcolor[HTML]{000026}{\cellcolor{gray!6}{Thailand}} & \textcolor[HTML]{407326}{\cellcolor{gray!6}{}} & \textcolor[HTML]{BF7300}{\cellcolor{gray!6}{Germany (WEST)}}\\
\textcolor[HTML]{739999}{} & \textcolor[HTML]{000026}{} & \textcolor[HTML]{407326}{} & \textcolor[HTML]{BF7300}{Denmark}\\
\addlinespace
\textcolor[HTML]{739999}{\cellcolor{gray!6}{}} & \textcolor[HTML]{000026}{\cellcolor{gray!6}{}} & \textcolor[HTML]{407326}{\cellcolor{gray!6}{}} & \textcolor[HTML]{BF7300}{\cellcolor{gray!6}{USA}}\\
\bottomrule
\end{tabular}
\end{table}

\begin{center}\includegraphics[width=0.85\linewidth]{final_report_files/figure-latex/unnamed-chunk-7-1} \end{center}

\hypertarget{societal-practices-and-values}{%
\subsection{Societal Practices and
Values}\label{societal-practices-and-values}}

For this section, since we consider nine societal practice vs.~value
concepts, so maintaining a significance level of \(\alpha = 0.05\)
increases the experimentwise error rate:
\(P(\text{Any False Positive}) = 1 - P(\text{No False Positives}) = 1 - 0.95^{9} \approx 0.3698\).
To remedy this issue, we employ a \emph{Bonferroni correction} to change
the significance threshold from \(\alpha = 0.05\) to
\(\alpha = \frac{0.05}{9} \approx 0.0056\).

\hypertarget{conclusions}{%
\section{Conclusions}\label{conclusions}}

\hypertarget{suggestions-for-further-research}{%
\section{Suggestions for Further
Research}\label{suggestions-for-further-research}}

\hypertarget{limitations}{%
\section{Limitations}\label{limitations}}

While promising, there are a few issues with the data we have to
consider. First, the number of observations: 62 observations is a rather
small dataset, which suggest that any analysis will likely be limited.
Also, the data silences are important and should not be overlooked. Why
were certain countries not surveyed?

\end{document}
