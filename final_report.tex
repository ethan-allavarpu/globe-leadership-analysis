% Options for packages loaded elsewhere
\PassOptionsToPackage{unicode}{hyperref}
\PassOptionsToPackage{hyphens}{url}
%
\documentclass[
]{article}
\title{Stats 140XP: Final Project}
\author{Ethan Allavarpu \(\cdot\) Raymond Bai \(\cdot\) Jaclyn Chiu
\(\cdot\) Ariel Chow \(\cdot\) Carlie Lin \(\cdot\) Dara
Tan \and \textbf{Explore the GLOBE}}
\date{1 December 2021}

\usepackage{amsmath,amssymb}
\usepackage{lmodern}
\usepackage{iftex}
\ifPDFTeX
  \usepackage[T1]{fontenc}
  \usepackage[utf8]{inputenc}
  \usepackage{textcomp} % provide euro and other symbols
\else % if luatex or xetex
  \usepackage{unicode-math}
  \defaultfontfeatures{Scale=MatchLowercase}
  \defaultfontfeatures[\rmfamily]{Ligatures=TeX,Scale=1}
\fi
% Use upquote if available, for straight quotes in verbatim environments
\IfFileExists{upquote.sty}{\usepackage{upquote}}{}
\IfFileExists{microtype.sty}{% use microtype if available
  \usepackage[]{microtype}
  \UseMicrotypeSet[protrusion]{basicmath} % disable protrusion for tt fonts
}{}
\makeatletter
\@ifundefined{KOMAClassName}{% if non-KOMA class
  \IfFileExists{parskip.sty}{%
    \usepackage{parskip}
  }{% else
    \setlength{\parindent}{0pt}
    \setlength{\parskip}{6pt plus 2pt minus 1pt}}
}{% if KOMA class
  \KOMAoptions{parskip=half}}
\makeatother
\usepackage{xcolor}
\IfFileExists{xurl.sty}{\usepackage{xurl}}{} % add URL line breaks if available
\IfFileExists{bookmark.sty}{\usepackage{bookmark}}{\usepackage{hyperref}}
\hypersetup{
  pdftitle={Stats 140XP: Final Project},
  pdfauthor={Ethan Allavarpu \textbackslash cdot Raymond Bai \textbackslash cdot Jaclyn Chiu \textbackslash cdot Ariel Chow \textbackslash cdot Carlie Lin \textbackslash cdot Dara Tan; },
  hidelinks,
  pdfcreator={LaTeX via pandoc}}
\urlstyle{same} % disable monospaced font for URLs
\usepackage[margin=1in]{geometry}
\usepackage{graphicx}
\makeatletter
\def\maxwidth{\ifdim\Gin@nat@width>\linewidth\linewidth\else\Gin@nat@width\fi}
\def\maxheight{\ifdim\Gin@nat@height>\textheight\textheight\else\Gin@nat@height\fi}
\makeatother
% Scale images if necessary, so that they will not overflow the page
% margins by default, and it is still possible to overwrite the defaults
% using explicit options in \includegraphics[width, height, ...]{}
\setkeys{Gin}{width=\maxwidth,height=\maxheight,keepaspectratio}
% Set default figure placement to htbp
\makeatletter
\def\fps@figure{htbp}
\makeatother
\setlength{\emergencystretch}{3em} % prevent overfull lines
\providecommand{\tightlist}{%
  \setlength{\itemsep}{0pt}\setlength{\parskip}{0pt}}
\setcounter{secnumdepth}{-\maxdimen} % remove section numbering
\usepackage{booktabs}
\usepackage{longtable}
\usepackage{array}
\usepackage{multirow}
\usepackage{wrapfig}
\usepackage{float}
\usepackage{colortbl}
\usepackage{pdflscape}
\usepackage{tabu}
\usepackage{threeparttable}
\usepackage{threeparttablex}
\usepackage[normalem]{ulem}
\usepackage{makecell}
\usepackage{xcolor}
\usepackage{booktabs}
\usepackage{longtable}
\usepackage{array}
\usepackage{multirow}
\usepackage{wrapfig}
\usepackage{float}
\usepackage{colortbl}
\usepackage{pdflscape}
\usepackage{tabu}
\usepackage{threeparttable}
\usepackage{threeparttablex}
\usepackage[normalem]{ulem}
\usepackage{makecell}
\usepackage{xcolor}
\ifLuaTeX
  \usepackage{selnolig}  % disable illegal ligatures
\fi

\begin{document}
\maketitle

{
\setcounter{tocdepth}{1}
\tableofcontents
}
\vfill

\newpage

\hypertarget{abstract}{%
\section{Abstract}\label{abstract}}

We wanted to answer two main questions: which leadership qualities do
countries tend to view similarly and if countries align their
perceptions of societal practices and values. For determining leadership
qualities and similar countries, we used principal component analysis
(PCA) then \(k\)-means clustering to create four ``clusters'' of
countries with similar leadership beliefs. We used the \_\_\_\_ method
to \_\_\_\_. We found that \_\_\_\_. We also looked at \_\_\_\_\_. In
the future, we recommend looking into \_\_\_\_\_. Some limitations to
our project are \_\_\_\_\_\_.

\hypertarget{problem-statements}{%
\section{Problem Statements}\label{problem-statements}}

\begin{enumerate}
\def\labelenumi{\arabic{enumi}.}
\tightlist
\item
  Which characteristics or traits do countries tend to group together
  when determining ``good'' leadership values?

  \begin{itemize}
  \tightlist
  \item
    Which countries have similar perceptions of these leadership values?
  \end{itemize}
\item
  Do societal practices and societal values align?

  \begin{itemize}
  \tightlist
  \item
    If they do not, which practices and values deviate most
    significantly?
  \end{itemize}
\end{enumerate}

\hypertarget{description-of-dataset}{%
\section{Description of Dataset}\label{description-of-dataset}}

The data set we have chosen to analyze is the \textbf{Dana Landis
Leadership} dataset, which comes from the GLOBE Research Survey. The
data provided in the folder had survey results for (1) leadership and
(2) societal and culture data and a PDF describing the nature of the
survey, but nothing more. To glean more information, we found the two
questionnaires (alpha and beta) described in the informational PDF to
get the original questions asked in the surveys. While we do not have a
``codebook'' in a traditional sense, the original questions asked may
help guide us in understanding what each variable means and how the
survey represents respondents answers numerically. The survey is on a 1
to 7 scale, with 1 being a negative response, 4 a ``neutral'' score, and
7 positive.

Here is a look at 6 full and complete observations from the leadership
survey:

Here is a look at 6 full and complete observations from the social and
cultural survey:

\hypertarget{visualization-and-exploratory-data-analysis}{%
\section{Visualization and Exploratory Data
Analysis}\label{visualization-and-exploratory-data-analysis}}

\hypertarget{the-data}{%
\section{The Data}\label{the-data}}

The data set we have chosen to analyze is the \textbf{Dana Landis
Leadership} dataset, which comes from the GLOBE Research Survey. The
data provided in the folder had survey results for (1) leadership and
(2) societal and culture data and a PDF describing the nature of the
survey, but nothing else. To glean more information, we found the two
questionnaires (alpha and beta) described in the informational PDF to
get the original questions. While we do not have a ``codebook'' in the
traditional sense, the original questions asked may help guide us in
understanding what each variable means and how the survey represents
respondents' answers numerically. The survey is on a 1 to 7 scale, with
1 being a negative response, 4 a ``neutral'' score, and 7 positive.

\hypertarget{leadership}{%
\section{Leadership}\label{leadership}}

\begin{center}\includegraphics[width=0.95\linewidth]{final_report_files/figure-latex/leadership-1} \end{center}

The above global heatmap presents the beliefs about whether autocracy
helps (higher values) or hinders (lower values) leaders in a given
country. Multiple similar graphics were generated for a variety of
leadership variables, ultimately leading us to recognize that different
regions of the world may favor certain leadership qualities over others
and that certain leadership qualities may be liked/disliked ``together''
(i.e., they have associations).

From boxplots we generated (not pictured), the difference by country
cluster (as defined by the dataset) seems visible, prompting us to
consider clustering countries to determine how they may be separated. We
do want to note that there are only 62 observations--this totals to
fewer than 62 countries because Germany and South Africa have two
observations each (West vs.~East and White vs.~Black, respecively). The
limited number of observations may limit us in the scope of our analysis
because we have less than a third of the total countries (demonstrated
by the vast swaths of grey on the world map for which there is no data)
and clustering would further reduce block/group sizes.

\hypertarget{societal-values-and-practices}{%
\section{Societal Values and
Practices}\label{societal-values-and-practices}}

\begin{center}\includegraphics[width=0.85\linewidth]{final_report_files/figure-latex/society-1} \end{center}

The plot grid demonstrates how, contrary to what \emph{should} happen,
the societal values of a country do not correlate well with the societal
practices (what \emph{should be} does not align with what \emph{is}).
This may be a source of further exploration, as we may want to see
whether there is a significant difference between the values and
practices. When we look at the per-region divisions though (by color),
the trend (or lack thereof) seems to persist for these concepts.

\hypertarget{analysis}{%
\section{Analysis}\label{analysis}}

\hypertarget{leadership-values}{%
\subsection{Leadership Values}\label{leadership-values}}

For the leadership values problem statement, the first objective is to
collapse the data into the first four principal components through
principal component analysis (PCA). PCA finds the directions which
capture the most variability (spread) in the data--the first four
account for the maximum variation. PCA allows us to visualize trends in
the leadership values: countries tend to have similar sentiments about
variables that ``point'' in the same direction (had principal component
values that aligned).

Before performing PCA, though, we remove the second-order factor
analysis variables due to the heavy correlation with the original
predictor variables and a more difficult interpretation of these
variables. Since our goal is to understand the relationship between
certain leadership characteristics, keeping these complicated variables
might reduce our understanding and make interpretation more difficult.

After performing PCA, we perform \(k\)-means clustering on the first
four principal components to determine the ``groups'' of leadership
characteristics that have similar perceptions. \(K\)-means clustering
creates \(k\) groups that are the ``most similar'' to the other
observations in their groups, minimizing the between-group variability:

\begin{center}\includegraphics[width=0.85\linewidth]{final_report_files/figure-latex/pca_heatmap-1} \end{center}

Here, we see three clusters: two large clusters and a group of two for
the other cluster. The green cluster seems to describe positive
characteristics (e.g., participative, inspirational), while the blue
cluster appears to describe negative characteristics (e.g., malevolent,
self-centered). From this, we note that these ``positive'' and
``negative'' characteristics tend to occur together. The third
``cluster'' (i.e., humane oriented and modesty) did not closely fit with
the other groups, leading us to believe that they may be considered
separately from the other characteristics.

After considering the leadership characteristics, we cluster countries
based on similar perceptions. We do this by using the variables of the
countries transformed into the first four principal components, then
running \(k\)-means clustering on those components. The clustering
results segregate the countries into the following segments:

\begin{center}\includegraphics[width=0.85\linewidth]{final_report_files/figure-latex/kmeans-1} \end{center}
\begin{table}[h]

\caption{\label{tab:country_table}Regions with Similar Leadership Perceptions}
\centering
\begin{tabular}[t]{>{}l>{}l>{}l>{}l}
\toprule
Cluster 1 & Cluster 2 & Cluster 3 & Cluster 4\\
\midrule
\textcolor[HTML]{739999}{\cellcolor{gray!6}{Argentina}} & \textcolor[HTML]{000026}{\cellcolor{gray!6}{Albania}} & \textcolor[HTML]{407326}{\cellcolor{gray!6}{France}} & \textcolor[HTML]{BF7300}{\cellcolor{gray!6}{Australia}}\\
\textcolor[HTML]{739999}{Bolivia} & \textcolor[HTML]{000026}{China} & \textcolor[HTML]{407326}{Morocco} & \textcolor[HTML]{BF7300}{Austria}\\
\textcolor[HTML]{739999}{\cellcolor{gray!6}{Brazil}} & \textcolor[HTML]{000026}{\cellcolor{gray!6}{Egypt}} & \textcolor[HTML]{407326}{\cellcolor{gray!6}{Qatar}} & \textcolor[HTML]{BF7300}{\cellcolor{gray!6}{Canada (English-speaking)}}\\
\textcolor[HTML]{739999}{Colombia} & \textcolor[HTML]{000026}{Georgia} & \textcolor[HTML]{407326}{South Africa (Black Sample)} & \textcolor[HTML]{BF7300}{Czech Republic}\\
\textcolor[HTML]{739999}{\cellcolor{gray!6}{Costa Rica}} & \textcolor[HTML]{000026}{\cellcolor{gray!6}{Hong Kong}} & \textcolor[HTML]{407326}{\cellcolor{gray!6}{}} & \textcolor[HTML]{BF7300}{\cellcolor{gray!6}{Denmark}}\\
\addlinespace
\textcolor[HTML]{739999}{Ecuador} & \textcolor[HTML]{000026}{India} & \textcolor[HTML]{407326}{} & \textcolor[HTML]{BF7300}{England}\\
\textcolor[HTML]{739999}{\cellcolor{gray!6}{El Salvador}} & \textcolor[HTML]{000026}{\cellcolor{gray!6}{Indonesia}} & \textcolor[HTML]{407326}{\cellcolor{gray!6}{}} & \textcolor[HTML]{BF7300}{\cellcolor{gray!6}{Finland}}\\
\textcolor[HTML]{739999}{Greece} & \textcolor[HTML]{000026}{Iran} & \textcolor[HTML]{407326}{} & \textcolor[HTML]{BF7300}{French Switzerland}\\
\textcolor[HTML]{739999}{\cellcolor{gray!6}{Guatemala}} & \textcolor[HTML]{000026}{\cellcolor{gray!6}{Japan}} & \textcolor[HTML]{407326}{\cellcolor{gray!6}{}} & \textcolor[HTML]{BF7300}{\cellcolor{gray!6}{Germany (EAST)}}\\
\textcolor[HTML]{739999}{Hungary} & \textcolor[HTML]{000026}{Kuwait} & \textcolor[HTML]{407326}{} & \textcolor[HTML]{BF7300}{Germany (WEST)}\\
\addlinespace
\textcolor[HTML]{739999}{\cellcolor{gray!6}{Israel}} & \textcolor[HTML]{000026}{\cellcolor{gray!6}{Malaysia}} & \textcolor[HTML]{407326}{\cellcolor{gray!6}{}} & \textcolor[HTML]{BF7300}{\cellcolor{gray!6}{Ireland}}\\
\textcolor[HTML]{739999}{Italy} & \textcolor[HTML]{000026}{Mexico} & \textcolor[HTML]{407326}{} & \textcolor[HTML]{BF7300}{Kazakhstan}\\
\textcolor[HTML]{739999}{\cellcolor{gray!6}{Philippines}} & \textcolor[HTML]{000026}{\cellcolor{gray!6}{Nigeria}} & \textcolor[HTML]{407326}{\cellcolor{gray!6}{}} & \textcolor[HTML]{BF7300}{\cellcolor{gray!6}{Namibia}}\\
\textcolor[HTML]{739999}{Slovenia} & \textcolor[HTML]{000026}{Poland} & \textcolor[HTML]{407326}{} & \textcolor[HTML]{BF7300}{Netherlands}\\
\textcolor[HTML]{739999}{\cellcolor{gray!6}{Spain}} & \textcolor[HTML]{000026}{\cellcolor{gray!6}{Russia}} & \textcolor[HTML]{407326}{\cellcolor{gray!6}{}} & \textcolor[HTML]{BF7300}{\cellcolor{gray!6}{New Zealand}}\\
\addlinespace
\textcolor[HTML]{739999}{Turkey} & \textcolor[HTML]{000026}{South Korea} & \textcolor[HTML]{407326}{} & \textcolor[HTML]{BF7300}{Portugal}\\
\textcolor[HTML]{739999}{\cellcolor{gray!6}{Zambia}} & \textcolor[HTML]{000026}{\cellcolor{gray!6}{Taiwan}} & \textcolor[HTML]{407326}{\cellcolor{gray!6}{}} & \textcolor[HTML]{BF7300}{\cellcolor{gray!6}{Singapore}}\\
\textcolor[HTML]{739999}{Zimbabwe} & \textcolor[HTML]{000026}{Thailand} & \textcolor[HTML]{407326}{} & \textcolor[HTML]{BF7300}{South Africa (White Sample)}\\
\textcolor[HTML]{739999}{\cellcolor{gray!6}{}} & \textcolor[HTML]{000026}{\cellcolor{gray!6}{Venezuela}} & \textcolor[HTML]{407326}{\cellcolor{gray!6}{}} & \textcolor[HTML]{BF7300}{\cellcolor{gray!6}{Sweden}}\\
\textcolor[HTML]{739999}{} & \textcolor[HTML]{000026}{} & \textcolor[HTML]{407326}{} & \textcolor[HTML]{BF7300}{Switzerland}\\
\addlinespace
\textcolor[HTML]{739999}{\cellcolor{gray!6}{}} & \textcolor[HTML]{000026}{\cellcolor{gray!6}{}} & \textcolor[HTML]{407326}{\cellcolor{gray!6}{}} & \textcolor[HTML]{BF7300}{\cellcolor{gray!6}{USA}}\\
\bottomrule
\end{tabular}
\end{table}

From the above grouping, we see that regionality (generally) determines
a country's respective cluster and leadership perspectives. Cluster 1
tends to describe Latin American and the Mediterranean, Cluster 2
generally includes Asia, and Cluster 4 describes the Anglo regions and
northern Europe. The one cluster that appears to be ``out there'' is
Cluster 3, which does not have a particular region associated with it.
To visualize the differences between these clusters, we create a heatmap
of the median variable values for each cluster:

\begin{center}\includegraphics[width=0.85\linewidth]{final_report_files/figure-latex/cluster_values-1} \end{center}

From here, we see that, despite the clustering, the countries tend to
agree on beneficial and detrimental leadership qualities; the main
difference is in the intensity: clusters 1 and 4 seem to have the
strongest responses (most positive and most negative), while cluster 3
does not seem to feel terribly strongly about positive leadership
characteristics. The few qualities of intrigue, though, are the ones for
which some clusters viewed positively (green) while others viewed
negatively (red):

\begin{itemize}
\tightlist
\item
  Status Conscious
\item
  Internally Competitive (Conflict Inducer)
\item
  Bureaucratic (Procedural)
\item
  Autonomous
\end{itemize}

\hypertarget{societal-practices-and-values}{%
\subsection{Societal Practices and
Values}\label{societal-practices-and-values}}

To determine whether societal practices and societal values align, we
first create scatterplots of societal values against societal practices
for each of the nine cultural dimensions:

\begin{center}\includegraphics[width=0.85\linewidth]{final_report_files/figure-latex/SPV Plots-1} \end{center}

From the scatterplots above, we note that societal practices and
societal values do not always align. For instance, in the case of the
Humane Orientation cultural dimension, there does not appear to be any
correlation between practices and values at all, with societal practices
in this dimension varying from less than 3.5 to more than 5.0 on the
survey's seven-point scale even though societal values were rated
relatively similarly across all of the surveyed countries. In some other
cases, such as for the Uncertainty Avoidance cultural dimension, there
appears to be a negative correlation between practices and values, as
indicated by the general trend of scores for practices in this dimension
decreasing even as scores for values increase. This negative correlation
might suggest that, rather than aligning, societal practices and
societal values are in fact in conflict.

To quantify the observations we made from our scatterplots above, we fit
simple linear regression models for each of the nine cultural
dimensions, using the societal values rating as the predictor and the
societal practices rating as the response. In other words, we fit the
model:

\[
y = \beta_0 + \beta_1 x
\]

Where \(x\) is the societal values rating and \(y\) is the societal
practices rating for each of the nine cultural dimensions. Additionally,
we performed \(t\)-tests for each dimension, to test the hypotheses
\(H_0: \beta_1 = 0\) and \(H_1: \beta_1 \ne 0\). For this, we employed
the \emph{Bonferroni correction} to change the significance threshold
from \(\alpha = 0.05\) to \(\alpha = \frac{0.05}{9} \approx 0.0056\)
since maintaining a significance level of \(\alpha = 0.05\) would
increase the experiment-wise error rate:
\(P(\text{Any False Positive}) = 1 - P(\text{No False Positives}) = 1 - 0.95^{9} \approx 0.3698\).
The table below shows the values of \(\beta_1\) that we obtained and the
corresponding p-values. Cultural dimensions for which the p-value is
lower than the corrected \(\alpha \approx 0.0056\) are indicated with
green shading.

\begin{table}[!h]

\caption{\label{tab:SPV SLR Table}Results of Simple Linear Regression by Cultural Dimension}
\centering
\begin{tabular}[t]{lrr}
\toprule
Cultural Dimension & Coefficient Value & p-value\\
\midrule
\cellcolor[HTML]{E5F5E0}{Uncertainty Avoidance} & \cellcolor[HTML]{E5F5E0}{-0.6199} & \cellcolor[HTML]{E5F5E0}{0.0000}\\
\cellcolor[HTML]{E5F5E0}{Institutional Collectivism} & \cellcolor[HTML]{E5F5E0}{-0.5251} & \cellcolor[HTML]{E5F5E0}{0.0000}\\
\cellcolor[HTML]{E5F5E0}{Power Distance} & \cellcolor[HTML]{E5F5E0}{-0.4991} & \cellcolor[HTML]{E5F5E0}{0.0006}\\
\cellcolor[HTML]{E5F5E0}{Future Orientation} & \cellcolor[HTML]{E5F5E0}{-0.4725} & \cellcolor[HTML]{E5F5E0}{0.0009}\\
Humane Orientation & -0.5944 & 0.0116\\
\addlinespace
\cellcolor[HTML]{F0F0F0}{Gender Egalitarianism} & \cellcolor[HTML]{F0F0F0}{0.2437} & \cellcolor[HTML]{F0F0F0}{0.0124}\\
Performance Orientation & -0.3459 & 0.0268\\
\cellcolor[HTML]{F0F0F0}{Assertiveness} & \cellcolor[HTML]{F0F0F0}{-0.1507} & \cellcolor[HTML]{F0F0F0}{0.0414}\\
In-group Collectivism & 0.4393 & 0.0991\\
\bottomrule
\end{tabular}
\end{table}

As shown above, most of the \(\beta_1\) values obtained were negative,
including all three values that were significant at the
\(\alpha = 0.56\)\% significant level as calculated using the Bonferroni
correction. This means that, as the rating for the societal values of a
cultural dimension increase, the rating for the societal practices of
that cultural dimension actually decrease. In other words, rather than
being aligned, societal practices in fact run counter to societal
values.

\hypertarget{conclusions}{%
\section{Conclusions}\label{conclusions}}

\hypertarget{suggestions-for-further-research}{%
\section{Suggestions for Further
Research}\label{suggestions-for-further-research}}

Regarding the leadership investigation, some further research we may
consider is investigating \emph{why} Cluster 3 differs so greatly from
the other three clusters. In Cluster 3 we have France, Qatar, Morocco,
and South Africa (Black Sample). Understanding why these countries are
so different may help us see if there is an underlying link. Similarly,
we may want to consider the differences between South Africa's Black and
White samples and Germany's East and West samples. Since the survey was
conducted in 2006, Apartheid and the Berlin Wall, respectively, may play
a role in any potential differences. Ultimately, the above two
suggestions relate to us considering the country's framework: looking at
other indices (such as a freedom index, government approval ratings, or
a happiness index) could shift us from an unsupervised learning
situation (just understanding relationships between variables) to a
model-building mindset in which we might determine which characteristics
seem to have associations with more freedom/approval/happiness.

For both leadership and the societal analyses, further research may be
conducting more surveys to look at the potential changes in ideology and
perspectives over time to understanding the changing dynamics at both
the national and global level.

\hypertarget{limitations}{%
\section{Limitations}\label{limitations}}

While promising, there are a few issues with the data we have to
consider. First, the number of observations: 62 observations is a rather
small dataset, which suggest that any analysis will likely be limited.
Also, the data silences are important and should not be overlooked. Why
were certain countries not surveyed?

\end{document}
